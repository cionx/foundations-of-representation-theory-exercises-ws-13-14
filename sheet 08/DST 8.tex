\documentclass[a4paper,10pt]{article}
%\documentclass[a4paper,10pt]{scrartcl}

\usepackage{xltxtra}
\usepackage{amsmath}
\usepackage{amssymb}
\usepackage{amsthm}
\usepackage{mathtools}
\usepackage{paralist}
\usepackage{leftidx}
\usepackage{tikz}

\theoremstyle{definition}
\newtheorem{own}{}
\newtheorem{lem}[own]{Lemma}
\newtheorem*{claim}{Claim}
\newtheorem{rem}[own]{Remark}
\newtheorem*{ia}{Base case}
\newtheorem*{is}{Induction step}

\newcommand{\N}{\operatorname{\mathbb{N}}}
\newcommand{\Z}{\operatorname{\mathbb{Z}}}
\newcommand{\Q}{\operatorname{\mathbb{Q}}}
\newcommand{\R}{\operatorname{\mathbb{R}}}
\newcommand{\C}{\operatorname{\mathbb{C}}}
\newcommand{\End}{\operatorname{End}}
\newcommand{\Hom}{\operatorname{Hom}}
\newcommand{\img}{\operatorname{img}}
\newcommand{\soc}{\operatorname{soc}}
\newcommand{\rad}{\operatorname{rad}}
\newcommand{\kchar}{\operatorname{char}}
\newcommand{\tr}{\operatorname{tr}}
\newcommand{\id}{\operatorname{id}}
\newcommand{\gen}[1]{\left\langle#1\right\rangle}
\newcommand{\vect}[1]{\begin{pmatrix}#1\end{pmatrix}}
\newcommand{\bvect}[1]{\begin{bmatrix}#1\end{bmatrix}}

\addtocounter{section}{28}

\renewcommand{\thesection}{Exercise \arabic{section}:}
\renewcommand{\thesubsection}{(\roman{subsection})}

\setromanfont[Mapping=tex-text]{Linux Libertine O}
% \setsansfont[Mapping=tex-text]{DejaVu Sans}
% \setmonofont[Mapping=tex-text]{DejaVu Sans Mono}

\parindent 0pt

\title{\textsc{Foundations of representation theory \\ \Large 8. Exercise sheet}}
\author{Jendrik Stelzner}
\date{\today}

\begin{document}
\maketitle





\section{}
We will assume that the vertices of $Q$ are ordered in the most obvious way. We define the subalgebra $B$ of $M_n(K)$ as
\[
 B := \{M = (m_{ij})_{ij} \in M_n(K) : m_{ij} = 0 \text{ for all } j > i \}.
\]
We will show that $KQ \cong B \cong A$.

For all $1 \leq i \leq j \leq n$  let $p_{ij}$ be the unique path in $Q$ from $i$ to $j$ and for all $1 \leq i, j \leq n$ let $E_{ij} \in M_n(K)$ be the matrix with $1$ as the $(i,j)$-entry and $0$ otherwise. ($E_{ij}$ maps $e_j$ to $e_i$.) We know that $(p_{ij})_{1 \leq i \leq j \leq n}$ is a basis of $KQ$, $(E_{ji})_{1 \leq i \leq j \leq n}$ is a basis of $B$ and $(E_{ij})_{1 \leq i \leq j \leq n}$ is a basis of $A$.

Let $\phi : KQ \rightarrow B$ be the linear map given by $\phi(p_{ij}) = E_{ji}$ for all $1 \leq i \leq j \leq n$. $\phi$ is a $K$-algebra homomorphism since for all $1 \leq i \leq j \leq n$ and $1 \leq k \leq l \leq n$
\begin{equation*}
 \phi(p_{ij} p_{kl})
 = \phi(\delta_{il} p_{kj})
 = \delta_{il} \phi(p_{kj})
 = \delta_{il} E_{jk}
 = E_{ji} E_{lk}
 = \phi(p_{ij}) \phi(p_{kl}).
\end{equation*}
$\phi$ is an isomorphism, because for the linear map $\psi: B \rightarrow KQ$ given by $\psi(E_{ji}) = p_{ij}$ for all $1 \leq i \leq j \leq n$ we have $\phi \psi = \id_B$ and $\psi \phi = \id_{KQ}$. Thus we have $KQ \cong B$.

To show that $B \cong A$ we notice that for the matrix
\[
 S :=
 \begin{pmatrix}
    &         & 1 \\
    & \diagup &   \\
  1 &         & 
 \end{pmatrix}
 \in M_n(K)
\]
with $S^2 = 1$ the map
\[
 \varphi : M_n(K) \rightarrow M_n(K), F \mapsto SFS
\]
is an vector space automorphism with $\varphi^2 = 1$. $\varphi$ is an algebra isomorphism because for all $F,G \in M_n(K)$
\[
 \varphi(FG) = SFGS = SFS^2GS = \varphi(F)\varphi(G).
\]
We also notice that $\varphi$ maps the Basis $(E_{ij})_{1 \leq i \leq j \leq n}$ of $A$ to the basis $(E_{ji})_{1 \leq i \leq j \leq n}$ of $B$, thus $\varphi_{|A} \rightarrow \varphi_{|B}$ is an algebra isomorphism.





\section{}
We name the vertices of $Q$ as $1$ and $2$ with $s(\alpha) = t(\alpha) = 1$ and the arrow from $1$ to $2$ as $p$. By definition
\[
 P := \{e_1, e_2, p\} \cup \bigcup_{n \geq 1} \{\alpha^n, p\alpha^n \}
\]
is a basis of $KQ$. It is clear that
\[
 B =
 \begin{pmatrix}
  K[T] & 0 \\
  K[T] & K
 \end{pmatrix}
\]
is a $K$-algebra via the usual matrix multiplication. We define the linear map $\phi : KQ \rightarrow B$ by
\begin{gather*}
 \phi(e_1) = \vect{1 & 0 \\ 0 & 0},
 \phi(e_2) = \vect{0 & 0 \\ 0 & 1},
 \phi(p) = \vect{0 & 0 \\ 1 & 0} \text{ and} \\
 \phi(\alpha^n) = \vect{T^n & 0 \\ 0 & 0},
 \phi(p\alpha^n) = \vect{0 & 0 \\ T^n & 0} \text{ for all } n \geq 1.
\end{gather*}
It is clear that $\phi$ induces a bijection between $P$ and a basis of $B$, so $\phi$ is a vector space isomorphism. It is also easy to see that $\phi$ is an algebra homomorphism, because $\phi(xy) = \phi(x)\phi(y)$ for all $x,y \in P$ (this can be directly shown by some boring matrix multiplication which I will not include here). Thus $KQ \cong B$.

The ideal $I = (\alpha^2)$ in $KQ$ generated by the path $\alpha^2$ corresponds to the ideal $J = (\phi(\alpha)^2)$ in $B$ generated by $\phi(\alpha)^2$. Because
\[
 \begin{pmatrix}
  a & 0 \\
  b & c
 \end{pmatrix}
 \underbrace{\begin{pmatrix}
  T^2 & 0 \\
  0 & 0
 \end{pmatrix}}_{= \phi(\alpha)^2}
 \begin{pmatrix}
  d & 0 \\
  e & f
 \end{pmatrix}
 =
 \begin{pmatrix}
  adT^2 & 0 \\
  bdT^2 & 0
 \end{pmatrix}
\]
we find that
\[
 J
 =
 (\phi(\alpha)^2)
 =
 \begin{pmatrix}
  (T^2) & 0 \\
  (T^2) & 0
 \end{pmatrix}.
\]
Thus $\phi$ induces an algebra isomorphism $\bar{\phi}$ between the algebras $KQ/I$ and $B/J = A$.





\section{}


\subsection*{$\neg$(ii) $\Rightarrow$ $\neg$(i)}
Let $Q_V^1, \ldots, Q_V^n$ be the (weakly) connected components of $Q_V$. By assumption $n \geq 2$. For $j=1,\ldots,n$ we define the non-zero representation $V^j = (V^j_i, V^j_a)_{i \in Q_0, a \in Q_1}$ of $Q$ as
\[
 V^j_i =
 \begin{cases}
  V_i & \text{if } i \in (Q^j_V)_0, \\
    0 & \text{if } i \not\in (Q^j_V)_0,
 \end{cases}
 \text{ and }
 V^j_a =
 \begin{cases}
  V_a & \text{if } a \in (Q^j_V)_1, \\
    0 & \text{if } a \not\in (Q^j_V)_1.
 \end{cases}
\]
From the definition of $Q_V$ and because $V$ is thin it directly follows that $V = \bigoplus_{j=1}^n V^j$. Thus $V$ is decomposable.


\subsection*{$\neg$(i) $\Rightarrow$ $\neg$(ii)}
Let $V = V^1 \oplus V^2$ with $V^1, V^2 \neq 0$. Because $V$ is thin it follows that for all $i,j \in Q_0$ with $V^1_i \neq 0$, $V^2_i = 0$ and $V^2_j \neq 0$, $V^1_j = 0$ the corresponding vertices $i$ and $j$ in $Q_V$ have no arrows between them: We find that for any $a \in Q_1$ from $i$ to $j$ or from $j$ to $i$ we have $V^1_a = 0$ and $V^2_a = 0$, thus $V_a = V^1_a \oplus V^2_a = 0$. So $Q_V$ has at least two connected components, one containing $i$ and one containing $j$.


\subsection*{(ii) $\Rightarrow$ (iii)}
Let $f \in \End_Q(V)$. It is clear that $f_i = 0$ for all $i \in Q_0 \setminus (Q_V)_0$. For all $i \in (Q_V)_0$ we have $\dim(V_i) = 1$ because $V$ is thin, and thus $f_i = \lambda_i 1_{V_i}$. Now let $i \in (Q_V)_0$ be fixed. Because $Q_V$ is connected we find $j \in (Q_V)_0$ s.t. an arrow $a$ from $i$ to $j$ or from $j$ to $i$ exists in $(Q_V)_1$. Thus we get one of the following commutative diagramms:
\begin{center}
\begin{minipage}{.3\textwidth}
 \begin{tikzpicture}[node distance=2cm, auto]
  \node (Vi) {$V_i$};
  \node (Vi2) [right of = Vi] {$V_i$};
  \node (Vj)  [below of = Vi] {$V_j$};
  \node (Vj2) [right of = Vj] {$V_j$};
  \draw[->] (Vi) to node {$\lambda_i$} (Vi2);
  \draw[->] (Vj) to node {$\lambda_j$} (Vj2);
  \draw[->] (Vi) to node {$V_a$} (Vj);
  \draw[->] (Vi2) to node {$V_a$} (Vj2);
 \end{tikzpicture}
\end{minipage}%
\begin{minipage}{.3\textwidth}
 \begin{tikzpicture}[node distance=2cm, auto]
  \node (Vi) {$V_i$};
  \node (Vi2) [right of = Vi] {$V_i$};
  \node (Vj)  [below of = Vi] {$V_j$};
  \node (Vj2) [right of = Vj] {$V_j$};
  \draw[->] (Vi) to node {$\lambda_i$} (Vi2);
  \draw[->] (Vj) to node {$\lambda_j$} (Vj2);
  \draw[->] (Vj) to node {$V_a$} (Vi);
  \draw[->] (Vj2) to node {$V_a$} (Vi2);
 \end{tikzpicture}
\end{minipage}%
\end{center}
Because $V_i$ and $V_j$ are one-dimensional and $V_a \neq 0$ we find that in both cases $\lambda_i = \lambda_j$. Because $Q_V$ is connected we find inductively that $\lambda_i = \lambda_j$ for all $j \in (Q_V)_0$. Thus we get $f_j = \lambda_i \id_{V_j}$ for all $j \in (Q_V)_0$ and therefore $f = \lambda_i 1_V$. It follows that $\End_K(Q) \cong K$.


\subsection*{(iii) $\Rightarrow$ (i)}
From $\End_Q(V) \cong K$ it directly follows that $V$ is indecomposable.










\end{document}
