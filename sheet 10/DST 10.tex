\documentclass[a4paper,10pt]{article}
%\documentclass[a4paper,10pt]{scrartcl}

\usepackage{xltxtra}
\usepackage{amsmath}
\usepackage{amssymb}
\usepackage{amsthm}
\usepackage{mathtools}
\usepackage{paralist}
\usepackage{leftidx}
\usepackage{tikz}

\theoremstyle{definition}
\newtheorem{own}{}
\newtheorem{lem}[own]{Lemma}
\newtheorem*{claim}{Claim}
\newtheorem{rem}[own]{Remark}
\newtheorem*{ia}{Base case}
\newtheorem*{is}{Induction step}

\newcommand{\N}{{\operatorname{\mathbb{N}}}}
\newcommand{\Z}{{\operatorname{\mathbb{Z}}}}
\newcommand{\Q}{{\operatorname{\mathbb{Q}}}}
\newcommand{\R}{{\operatorname{\mathbb{R}}}}
\newcommand{\C}{{\operatorname{\mathbb{C}}}}
\newcommand{\End}{\operatorname{End}}
\newcommand{\Hom}{\operatorname{Hom}}
\newcommand{\img}{\operatorname{im}}
\newcommand{\soc}{\operatorname{soc}}
\newcommand{\rad}{\operatorname{rad}}
\newcommand{\kchar}{\operatorname{char}}
\newcommand{\tr}{\operatorname{tr}}
\newcommand{\id}{\operatorname{id}}
\newcommand{\gen}[1]{\left\langle#1\right\rangle}
\newcommand{\vect}[1]{\begin{pmatrix}#1\end{pmatrix}}
\newcommand{\bvect}[1]{\begin{bmatrix}#1\end{bmatrix}}
\newcommand{\li}[2]{\leftidx{_{#1}}{#2}}

\addtocounter{section}{36}

\renewcommand{\thesection}{Exercise \arabic{section}:}
\renewcommand{\thesubsection}{(\roman{subsection})}

\setromanfont[Mapping=tex-text]{Linux Libertine O}
% \setsansfont[Mapping=tex-text]{DejaVu Sans}
% \setmonofont[Mapping=tex-text]{DejaVu Sans Mono}

\parindent 0pt

\title{\textsc{Foundations of representation theory \\ \Large 10. Exercise sheet}}
\author{Jendrik Stelzner}
\date{\today}

\begin{document}
\maketitle





\addtocounter{section}{1}





\section{}


\subsection{}
Let $x \in V_3$ with $a_3(x) = 0$. Because the diagram commutes we find that
\[
 0 = g_3 a_3(x) = a_4 f_3(x).
\]
Because $a_4$ is a monomorphism this means that $f_3(x) = 0$. From the exactness of the upper row we get that $x \in \ker f_3 = \img f_2$, so there exists $y \in V_2$ with $f_2(y) = x$. Using the commutativity of the diagram we get that
\[
 0 = a_3(x) = a_3 f_2(y) = g_2 a_2(y).
\]
Therefor $a_2(y) \in \ker g_2 = \img g_1 = \img (g_1 a_1)$, whereby we used that $a_1$ is an epimorphism. So we find some $z \in V_1$ with $g_1 a_1(z) = a_2(y)$. Combining all of the above and using the commutativity of the diagram we find that
\[
 a_2( f_1(z) ) = a_2 f_1(z) = g_1 a_1(z) = a_2(y).
\]
Because $a_2$ is a monomorphism it follows that $f_1(z) = y$. Because the upper row is exact we get
\[
 x = f_2(y) = f_2 f_1(z) = 0.
\]
So $a_3$ is a momomorphism.


\subsection{}
Let $x \in W_3$. We look at $g_3(x) \in W_4$. Because $a_4$ is an epimorphism there exists some $y \in V_4$ with $a_4(y) = g_3(x)$. We notice that $f_4(y) = 0$: Using the exactness and commutativity of the diagram we get
\[
 a_5 f_4(y) = g_4 a_4(y) = g_4 g_3(x) = 0,
\]
and because $a_5$ is a monomorphism it follows that $f_4(y) = 0$. Because the upper row is exact we have
\[
 y \in \ker f_4 = \img f_3,
\]
and therefore there exists some $z \in V_3$ with $f_3(z) = y$.

We now look at $a_3(z)$: From the above and the commutativity of the diagram we get that
\[
 g_3(x) = a_4(y) = a_4 f_3(z) = g_3 a_3(z) = g_3( a_3(z) ),
\]
so
\[
 g_3( x-a_3(z) ) = 0.
\]
Because the lower row is exact and $a_2$ is an epimorphism and the diagram commutes it follows that
\[
 x - a_3(z) \in \ker g_3 = \img g_2 = \img (g_2 a_2) = \img (a_3 f_2) \subseteq \img a_3.
\]
Because $a_3(z) \in \img a_3$ this gives us $x \in \img a_3$. So $a_3$ is an epimorphism.


\subsection{}
It is enough to assume that $a_1$ is an epimorphism, $a_5$ is a monomorphism and $a_2$ and $a_4$ are isomorphisms. Combining the two previous statements it then directly follows that $a_3$ is a mono- and an endomorphism, and therefore an isomorphism.





\addtocounter{section}{1}





\section{}
We define the $K$-vector space
\[
 K^\N := \{ (\lambda_i)_{i \in \N} : \lambda_i \in K \text{ for all } i \in \N \}
\]
and for all $n \in \N$ the subspaces
\[
 K^\N_n := \{(\lambda_i)_{i \in \N} : \lambda_0 = \ldots = \lambda_{n-1} = 0\}.
\]
and
\[
 K_n := \{(\lambda_i)_{i \in \N} : \lambda_i = 0 \text{ for all } i \geq n\}
\]
It is clear that for all $n \in \N$ we have $K^\N = K_n \oplus K^\N_n$ and $K_n \cong K^n$.

We now look at the short exact sequence sequence
\begin{center}
 \begin{tikzpicture}[node distance = 1.5cm]
  \node (0_1)                      {$0$};
  \node (K)     [right of = 0_1]   {$K_1$};
  \node (K^N)   [right of = K]     {$K^\N$};
  \node (K^N_2) [right of = K^N]   {$K^\N_2$};
  \node (0_2)   [right of = K^N_2] {$0$};
  \draw[->] (0_1) to (K);
  \draw[->] (K) to node[above] {$f$} (K^N);
  \draw[->] (K^N) to node[above] {$g$} (K^N_2);
  \draw[->] (K^N_2) to (0_2);
 \end{tikzpicture}
\end{center}
where $f$ is the canonical inclusion and
\[
 g: K^\N \to K^\N_2, (\lambda_0, \lambda_1, \lambda_2, \ldots) \mapsto (0,0,\lambda_1, \lambda_2, \ldots).
\]
Now we just use the direct sum decomposition $K^\N = K_2 \oplus K^\N_2$, where it is clear that $K_1 \ncong K_2$.





\end{document}
