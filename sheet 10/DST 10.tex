\documentclass[a4paper,10pt]{article}
%\documentclass[a4paper,10pt]{scrartcl}

\usepackage{xltxtra}
\usepackage{amsmath}
\usepackage{amssymb}
\usepackage{amsthm}
\usepackage{mathtools}
\usepackage{paralist}
\usepackage{leftidx}
\usepackage{tikz}

\theoremstyle{definition}
\newtheorem{own}{}
\newtheorem{lem}[own]{Lemma}
\newtheorem*{claim}{Claim}
\newtheorem{rem}[own]{Remark}
\newtheorem*{ia}{Base case}
\newtheorem*{is}{Induction step}

\newcommand{\N}{{\operatorname{\mathbb{N}}}}
\newcommand{\Z}{{\operatorname{\mathbb{Z}}}}
\newcommand{\Q}{{\operatorname{\mathbb{Q}}}}
\newcommand{\R}{{\operatorname{\mathbb{R}}}}
\newcommand{\C}{{\operatorname{\mathbb{C}}}}
\newcommand{\End}{\operatorname{End}}
\newcommand{\Hom}{\operatorname{Hom}}
\newcommand{\img}{\operatorname{im}}
\newcommand{\soc}{\operatorname{soc}}
\newcommand{\rad}{\operatorname{rad}}
\newcommand{\kchar}{\operatorname{char}}
\newcommand{\tr}{\operatorname{tr}}
\newcommand{\id}{\operatorname{id}}
\newcommand{\gen}[1]{\left\langle#1\right\rangle}
\newcommand{\vect}[1]{\begin{pmatrix}#1\end{pmatrix}}
\newcommand{\bvect}[1]{\begin{bmatrix}#1\end{bmatrix}}
\newcommand{\li}[2]{\leftidx{_{#1}}{#2}}

\addtocounter{section}{36}

\renewcommand{\thesection}{Exercise \arabic{section}:}
\renewcommand{\thesubsection}{(\roman{subsection})}

\setromanfont[Mapping=tex-text]{Linux Libertine O}
% \setsansfont[Mapping=tex-text]{DejaVu Sans}
% \setmonofont[Mapping=tex-text]{DejaVu Sans Mono}

\parindent 0pt

\title{\textsc{Foundations of representation theory \\ \Large 10. Exercise sheet}}
\author{Jendrik Stelzner}
\date{\today}

\begin{document}
\maketitle





\addtocounter{section}{3}





\section{}
We define the $K$-vector space
\[
 K^\N := \{ (\lambda_i)_{i \in \N} : \lambda_n \in K \text{ for all } n \in \N \}
\]
and for all $n \in \N$ the $K$-vector space
\[
 K^\N_n := \{(\lambda_i)_{i \in \N} : \lambda_0 = \ldots = \lambda_{n-1} = 0\}.1
\]
In particular $K^\N_0 = K^\N$.
We now look at the short exact sequence sequence
\begin{center}
 \begin{tikzpicture}[node distance = 1.5cm]
  \node (0_1)                      {$0$};
  \node (K)     [right of = 0_1]   {$K$};
  \node (K^N)   [right of = K]     {$K^\N$};
  \node (K^N_2) [right of = K^N]   {$K^\N_2$};
  \node (0_2)   [right of = K^N_2] {$0$};
  \draw[->] (0_1) to (K);
  \draw[->] (K) to node[above] {$f$} (K^N);
  \draw[->] (K^N) to node[above] {$g$} (K^N_2);
  \draw[->] (K^N_2) to (0_2);
 \end{tikzpicture}
\end{center}
given by the $K$-linear maps
\begin{gather*}
 f: K \rightarrow K^\N, \lambda \mapsto (\lambda, 0, 0, \ldots)
\shortintertext{and}
 g: K^\N \rightarrow K^\N_2, (\lambda_0, \lambda_1, \lambda_2, \ldots) \mapsto (0,0,\lambda_1, \lambda_2, \ldots).
\end{gather*}
Now we just use the direct sum decomposition $K^\N = K^2 \oplus K^\N_2$, where it is clear that $K \ncong K^2$.













\end{document}
