\documentclass[a4paper,10pt]{article}
%\documentclass[a4paper,10pt]{scrartcl}

\usepackage{xltxtra}
\usepackage{amsmath}
\usepackage{amssymb}
\usepackage{amsthm}
\usepackage{mathtools}
\usepackage{paralist}

\theoremstyle{definition}
\newtheorem{own}{}
\newtheorem{lem}[own]{Lemma}
\newtheorem*{claim}{Claim}
\newtheorem{rem}[own]{Remark}
\newtheorem*{ia}{Base case}
\newtheorem*{is}{Induction step}

\newcommand{\N}{\operatorname{\mathbb{N}}}
\newcommand{\Z}{\operatorname{\mathbb{Z}}}
\newcommand{\Q}{\operatorname{\mathbb{Q}}}
\newcommand{\R}{\operatorname{\mathbb{R}}}
\newcommand{\C}{\operatorname{\mathbb{C}}}
\newcommand{\End}{\operatorname{End}}
\newcommand{\Hom}{\operatorname{Hom}}
\newcommand{\img}{\operatorname{img}}
\newcommand{\soc}{\operatorname{soc}}
\newcommand{\rad}{\operatorname{rad}}
\newcommand{\kchar}{\operatorname{char}}
\newcommand{\tr}{\operatorname{tr}}
\newcommand{\id}{\operatorname{id}}
\newcommand{\gen}[1]{\left\langle#1\right\rangle}
\newcommand{\vect}[1]{\begin{pmatrix}#1\end{pmatrix}}

\addtocounter{section}{20}

\renewcommand{\thesection}{Exercise \arabic{section}:}
\renewcommand{\thesubsection}{(\roman{subsection})}

\setromanfont[Mapping=tex-text]{Linux Libertine O}
% \setsansfont[Mapping=tex-text]{DejaVu Sans}
% \setmonofont[Mapping=tex-text]{DejaVu Sans Mono}

\parindent 0pt

\title{\textsc{Foundations of representation theory \\ \Large 6. Exercise sheet}}
\author{Jendrik Stelzner}
\date{\today}

\begin{document}
\maketitle





\section{}
We will assume that $V$ is an artinian module.
$V$ is uniform, because $S \neq 0$ is contained in every non-zero submodule of $V$. This implies that $V$ is indecomposable.

For all $f \in \End(V)$ we have $\img f_{|S} \subseteq S$: If $f_{|S} = 0$ this is trivial. Otherwise $\img f_{|S} \subseteq V$ is a non-zero submodule, so $S \subseteq \img f_{|S}$. Because $S$ is non-zero, $f^{-1}_{|S}(S) \subseteq S$ is a non-zero submodule. Because $S$ is simple we get $S = f^{-1}_{|S} (S)$ and thus $\img f_{|S} = S$.

This allows us to define $\varphi : \End(V) \rightarrow \End(S), f \mapsto f_{|S}$. It is obvious that $\varphi$ is a ring homomorphism. By assumption $\varphi$ is surjective. We know show that \[ \ker \varphi = \{f \in \End(V) : \text{$f$ is not invertible}\}. \]

It is clear that \[\ker \varphi \subseteq \{f \in \End(V) : \text{$f$ is not invertible}\}.\] Let $f \in \End(V)$ be not injective. Because $\ker f \neq 0$ is a submodule we have $S \subseteq \ker f$, so $f_{|S} = 0$. Let $g \in \End(V)$ be injective but not surjective. We get a descending chain
\[
 V \supseteq \img g \supseteq \img g^2 \supseteq \img g^3 \supseteq \ldots
\]
of submodules of $V$. By assumption this chain eventually stabilizes, i.e. there exists some $N \in \N$ with $\img g^n = \img g^{n+1}$ for all $n \geq N$. Because $g$ is injective we also have $\ker g^n = \ker g^{n+1}$ for all $n \in \N$. This implies that
\[
 V = \ker g^N \oplus \img g^N.
\]
Because $V$ is indecomposable this implies that $\ker g^n = 0$ and $\img g^N = V$ or $\ker g^N = V$ and $\img g^N = 0$. So $g$ is either not injective or surjective, which is contradicts either the injectivity or non-surjectivity of $g$. So $g$ has to be non-injective and thus contained in $\ker \varphi$.

Because
\[
 \ker \varphi = \{f \in \End(V) : \text{$f$ is not invertible}\}
\]
is an ideal in $\End(V)$, we get that $\End(V)$ is local. so
\[
 J(\End(V)) = \{f \in \End(V) : \text{$f$ is not invertible}\} = \ker \varphi
\]
and thus
\[
 \End(V) / J(\End(V)) = \End(V) / \ker \varphi \cong \img \varphi = \End(S).
\]













\end{document}
