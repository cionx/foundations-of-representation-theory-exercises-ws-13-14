\documentclass[a4paper,10pt]{article}
%\documentclass[a4paper,10pt]{scrartcl}

\usepackage{xltxtra}
\usepackage{amsmath}
\usepackage{amssymb}
\usepackage{amsthm}
\usepackage{mathtools}
\usepackage{paralist}
\usepackage{leftidx}

\theoremstyle{definition}
\newtheorem{own}{}
\newtheorem{lem}[own]{Lemma}
\newtheorem*{claim}{Claim}
\newtheorem{rem}[own]{Remark}
\newtheorem*{ia}{Base case}
\newtheorem*{is}{Induction step}

\newcommand{\N}{\operatorname{\mathbb{N}}}
\newcommand{\Z}{\operatorname{\mathbb{Z}}}
\newcommand{\Q}{\operatorname{\mathbb{Q}}}
\newcommand{\R}{\operatorname{\mathbb{R}}}
\newcommand{\C}{\operatorname{\mathbb{C}}}
\newcommand{\End}{\operatorname{End}}
\newcommand{\Hom}{\operatorname{Hom}}
\newcommand{\img}{\operatorname{img}}
\newcommand{\soc}{\operatorname{soc}}
\newcommand{\rad}{\operatorname{rad}}
\newcommand{\kchar}{\operatorname{char}}
\newcommand{\tr}{\operatorname{tr}}
\newcommand{\id}{\operatorname{id}}
\newcommand{\gen}[1]{\left\langle#1\right\rangle}
\newcommand{\vect}[1]{\begin{pmatrix}#1\end{pmatrix}}
\newcommand{\bvect}[1]{\begin{bmatrix}#1\end{bmatrix}}

\addtocounter{section}{20}

\renewcommand{\thesection}{Exercise \arabic{section}:}
\renewcommand{\thesubsection}{(\roman{subsection})}

\setromanfont[Mapping=tex-text]{Linux Libertine O}
% \setsansfont[Mapping=tex-text]{DejaVu Sans}
% \setmonofont[Mapping=tex-text]{DejaVu Sans Mono}

\parindent 0pt

\title{\textsc{Foundations of representation theory \\ \Large 6. Exercise sheet}}
\author{Jendrik Stelzner}
\date{\today}

\begin{document}
\maketitle





\section{}
We will assume that $V$ is an artinian module.
$V$ is uniform, because $S \neq 0$ is contained in every non-zero submodule of $V$. This implies that $V$ is indecomposable.

For all $f \in \End(V)$ we have $\img f_{|S} \subseteq S$: If $f_{|S} = 0$ this is trivial. Otherwise $\img f_{|S} \subseteq V$ is a non-zero submodule, so $S \subseteq \img f_{|S}$. Because $S$ is non-zero, $f^{-1}_{|S}(S) \subseteq S$ is a non-zero submodule. Because $S$ is simple we get $S = f^{-1}_{|S} (S)$ and thus $\img f_{|S} = S$.

This allows us to define $\varphi : \End(V) \rightarrow \End(S), f \mapsto f_{|S}$. It is obvious that $\varphi$ is a ring homomorphism. By assumption $\varphi$ is surjective. We know show that \[ \ker \varphi = \{f \in \End(V) : \text{$f$ is not invertible}\}. \]

It is clear that \[\ker \varphi \subseteq \{f \in \End(V) : \text{$f$ is not invertible}\}.\] Let $f \in \End(V)$ be not injective. Because $\ker f \neq 0$ is a submodule we have $S \subseteq \ker f$, so $f_{|S} = 0$. Let $g \in \End(V)$ be injective but not surjective. We get a descending chain
\[
 V \supseteq \img g \supseteq \img g^2 \supseteq \img g^3 \supseteq \ldots
\]
of submodules of $V$. By assumption this chain eventually stabilizes, i.e. there exists some $N \in \N$ with $\img g^n = \img g^{n+1}$ for all $n \geq N$. Because $g$ is injective we also have $\ker g^n = \ker g^{n+1}$ for all $n \in \N$. This implies that
\[
 V = \ker g^N \oplus \img g^N.
\]
Because $V$ is indecomposable this implies that $\ker g^n = 0$ and $\img g^N = V$ or $\ker g^N = V$ and $\img g^N = 0$. So $g$ is either not injective or surjective, which is contradicts either the injectivity or non-surjectivity of $g$. So $g$ has to be non-injective and thus contained in $\ker \varphi$.

Because
\[
 \ker \varphi = \{f \in \End(V) : \text{$f$ is not invertible}\}
\]
is an ideal in $\End(V)$, we get that $\End(V)$ is local. so
\[
 J(\End(V)) = \{f \in \End(V) : \text{$f$ is not invertible}\} = \ker \varphi
\]
and thus
\[
 \End(V) / J(\End(V)) = \End(V) / \ker \varphi \cong \img \varphi = \End(S).
\]





\section{}


\subsection{}
\begin{rem}\label{rem: endomorphisms K[T]}
 Every endomorphism $h$ of the module $V = (K[T], T \cdot)$ is of the form
 \[
  h : f \mapsto p \cdot f
 \]
 for some polynomial $p \in K[T]$, and each such map is an endomorphism von $V$.
\end{rem}
\begin{proof}
 It is obvious that every such map is an module endomorphism of $V$. To show that every endomorphism is of this form we denote $T \cdot$ as $\phi : f \mapsto T \cdot f$ and $p := h(1)$. Because $h$ is a module homomorphism, it follows that for all $\sum_{i=0}^n a_i T^i \in K[T]$
 \begin{align*}
  h\left( \sum_{i=0}^n a_i T^i \right)
  &= h\left( \sum_{i=0}^n a_i \phi^i(1) \right)
  = \sum_{i=0}^n a_i \phi^i (h(1)) \\
  &= \sum_{i=0}^n a_i \phi^i(p)
  = \sum_{i=0}^n a_i T^i p
  = p \sum_{i=0}^n a_i T^i.
 \end{align*}
\end{proof}

We assume that the greatest common divisor of some polynomial $p \in K[T]$ and $0 \in K[T]$ is defined as $p$, because otherwise the statement does not hold true. In this case the direct summands $V \oplus 0$ and $0 \oplus V$ are counterexamples.

Let $U$ be a direct summand of $V \oplus V$ with $U \not\in \{0, V \oplus V\}$. We know that $U$ is the kernel of some idempotent endomorphism of $V$. Let $h \in \End(V)$ be idempotent with $\ker h = U$. We know that $h$ can be uniquely written as
\[
 h =
 \begin{bmatrix}
  h_1 & h_2 \\
  h_3 & h_4
 \end{bmatrix}
\]
with $h_1, h_2, h_3, h_4 \in \End(V)$. From remark \ref{rem: endomorphisms K[T]} it follows that there exists polynomials $p_1, p_2, p_3, p_4 \in K[T]$ such that
\[
 h =
 \begin{bmatrix}
  p_1 & p_2 \\
  p_3 & p_4
 \end{bmatrix}.
\]
We get that
\[
 \ker h = \left\{ \bvect{p\\q} \in V \oplus V : 
 \begin{bmatrix}
  p_1 & p_2 \\
  p_3 & p_4
 \end{bmatrix}
 \bvect{p\\q}
 = 0.
 \right\}
\]

We notice that the matrix
\[
 \begin{bmatrix}
  p_1 & p_2 \\
  p_3 & p_4
 \end{bmatrix}
\]
also describes an idempotent vector space endomorphism $h'$ of $K(X)^2$ with respect to the canonical basis, where $K(T)$ is the field of rational functions over $K$. It is clear that
\[
 \ker h = \ker h' \cap K[T],
\]
and that $\ker h'$ is a subspace of $K(T)^2$. So to show that $U$ is of the form $U_{f,g}$ for some polynomials $f,g$ with greatest common divisor $1$, it is enough to show that $\ker h'$ has a basis $\ltrans{\bvect{f & g}}$ with such polynomials $f$ and $g$. Because $U \not\in \{0, V \oplus V\}$ we know that $h$ is non-zero and no isomorphism, so $\ker h'$ is one-dimensional. Let $b = \ltrans{\bvect{f & g}}$ be a basis vector of $\ker h'$. We can assume that $f$ and $g$ are polynomials, because otherwise we can multiply $b$ by some non-zero scalar $r \in K[T] \subseteq K(T)$ such that $rf$ and $rg$ are polynomials. We can also assume that the greatest common divisor of $f$ and $g$ is $1$, because otherwise we can multiply $b$ with some scalar $r' \in K(T)$ such that the greatest common divisor of $r'f$ and $r'g$ is $1$. Because $\ltrans{\bvect{f & g}}$ is a basis vector of $\ker h$ it follows that
\[
 U = \ker h = \ker h' \cap K[T] = \{(hf, hg) : h \in K[T]\} = U_{f,g}.
\]

Now we show the $U_{f,g}$ are direct summands: It is clear that $U_{f,g}$ is a submodule of $V \oplus V$ for alle $f,g \in K[T]$. Assume that the greatest common divisor of $f$ and $g$ is $1$. If $f=0$, then it follows that $g=1$, so
\[
 U_{f,g} = 0 \oplus V = \ker \bvect{1&0\\1&0},
\]
and if $g=0$ it follows that $f=1$, so
\[
 U_{f,g} = V \oplus 0 = \ker \bvect{0&1\\0&1}.
\]
In both cases the endomorphisms are obvioulsy idempotent.
If $f,g \neq 0$ we know from linear algebra that we can find polynomials $r,s\in K[T]$ with $rf + sg = 1$. Then the endomorphism of $V \oplus V$ of the form
\[
 h = 
 \begin{bmatrix}
   sg & -sf \\
  -rg &  rf
 \end{bmatrix}
\]
with
\begin{align*}
 \begin{bmatrix}
   sg & -sf \\
  -rg &  rf
 \end{bmatrix}^2
 &=
 \begin{bmatrix}
  rsfg + s^2g^2 & -rsf^2-s^2fg \\
 -r^2fg - rsg^2 & r^2 f^2 + rsfg
 \end{bmatrix} \\
 &=
 \begin{bmatrix}
  sg(rf+sg) & -sf(rf+sg) \\
 -rg(rf+sg) & rf(rf+sg)
 \end{bmatrix}
 =
 \begin{bmatrix}
   sg & -sf \\
  -rg &  rf
 \end{bmatrix}
\end{align*}
is idempotent, and thus
\[
 \ker h = U_{f,g}
\]
is a direct summand of $V \oplus V$.


\subsection{}
We know that $U_{T,T-1}$ is a direct summand of $V \oplus V$, because the greatest common divisor of $T$ and $T-1$ is $1$. Because $U_{T,T-1} \neq 0$ and $U_{T,T-1} \neq V \oplus V$ we know that neither $0$ nor $V \oplus V$ is a direct complement of $U_{T,T-1}$ in $V \oplus V$. $U_{1,0}$ and $U_{0,1}$ are also no direct complements of $U_{T,T-1}$ in $V \oplus V$, because $(0,1) \not\in U_{T,T-1} + U_{1,0}$ and $(1,0) \not\in U_{T,T-1} + U_{0,1}$.





\section{}


\subsection*{$\soc(M(w))$}
Let $b_1, \ldots, b_{n+1}$ be the standard basis of $M(w)$. For all $1 \leq i \leq n+1$ let
\[
 h_x(b_i) := \min \{k \in \N : \phi^k_x(b_i) = 0\} \text{ and }
 h_y(b_i) = \min \{k \in \N : \phi^k_y(b_i) = 0\}.
\]
Let $c_1, \ldots, c_m \in \{b_1, \ldots, b_{n+1}\}$ be the pairwise different $b_i$ such that $h_x(b_i) = h_y(b_i) = 1$.

For every $v \in M(w)$ with $h_x(v) = h_y(v) = 1$ the subspace $\gen{v} = K v$ is a simple submodule of $M(w)$, because $\gen{x}$ is one-dimensional and $\phi_x(v) = \phi_y(v) = 0$.

Every non-zero submodule $U \subseteq M(w)$ contains such a simple module: Let $v \in U$ with $v \neq 0$. If $h_x(v) = h_y(v) = 1$, $\gen{v}$ is such a submodule. Otherwise $h_x(v) > 1$ or $h_x(v) > 1$. In the first case $\gen{\phi^{h_x(v)-1}_x(v)}$ is such a submodule, in the second case $\gen{\phi^{h_y(v)-1}_y(v)}$ is such a submodule.

It follows that every simple submodule of $M(w)$ is of the form $\gen{v}$ for some $v \in M(w)$ with $h_x(v) = h_y(v) = 1$. It is easy so see that each such $v$ is a linear combination of the $c_i$: For $v = \sum_{i=1}^{n+1} \lambda_i b_i$ we get
\[
 0 = \phi_x(v) = \sum_{i=1}^{n+1} \lambda_i \phi_x(b_i)
 \text{ and }
 0 = \phi_y(v) = \sum_{i=1}^{n+1} \lambda_i \phi_y(b_i),
\]
so for all $i$ we have the implication $\lambda_i \neq 0 \Rightarrow \phi_x(b_i) = \phi_y(b_i) = 1$. So we get
\[
 \soc(M(w))
 = \sum_{\substack{S \subseteq M(w) \\ S \text{ simple}}} S
 = \bigoplus_{i=1}^m \gen{c_i}.
\]
















\end{document}
