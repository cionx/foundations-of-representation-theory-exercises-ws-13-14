\documentclass[a4paper,10pt]{article}
%\documentclass[a4paper,10pt]{scrartcl}

\usepackage{xltxtra}
\usepackage{amsmath}
\usepackage{amssymb}
\usepackage{amsthm}
\usepackage{mathtools}
\usepackage{paralist}
\usepackage{leftidx}
\usepackage{tikz}

\theoremstyle{definition}
\newtheorem{own}{}
\newtheorem{lem}[own]{Lemma}
\newtheorem*{claim}{Claim}
\newtheorem{rem}[own]{Remark}
\newtheorem*{ia}{Base case}
\newtheorem*{is}{Induction step}

\newcommand{\N}{\operatorname{\mathbb{N}}}
\newcommand{\Z}{\operatorname{\mathbb{Z}}}
\newcommand{\Q}{\operatorname{\mathbb{Q}}}
\newcommand{\R}{\operatorname{\mathbb{R}}}
\newcommand{\C}{\operatorname{\mathbb{C}}}
\newcommand{\End}{\operatorname{End}}
\newcommand{\Hom}{\operatorname{Hom}}
\newcommand{\img}{\operatorname{im}}
\newcommand{\soc}{\operatorname{soc}}
\newcommand{\rad}{\operatorname{rad}}
\newcommand{\kchar}{\operatorname{char}}
\newcommand{\tr}{\operatorname{tr}}
\newcommand{\id}{\operatorname{id}}
\newcommand{\gen}[1]{\left\langle#1\right\rangle}
\newcommand{\vect}[1]{\begin{pmatrix}#1\end{pmatrix}}
\newcommand{\bvect}[1]{\begin{bmatrix}#1\end{bmatrix}}
\newcommand{\li}[2]{\leftidx{_{#1}}{#2}}

\addtocounter{section}{32}

\renewcommand{\thesection}{Exercise \arabic{section}:}
\renewcommand{\thesubsection}{(\roman{subsection})}

\setromanfont[Mapping=tex-text]{Linux Libertine O}
% \setsansfont[Mapping=tex-text]{DejaVu Sans}
% \setmonofont[Mapping=tex-text]{DejaVu Sans Mono}

\parindent 0pt

\title{\textsc{Foundations of representation theory \\ \Large 9. Exercise sheet}}
\author{Jendrik Stelzner}
\date{\today}

\begin{document}
\maketitle





\section{}
Assume that $\li{A}{A} \cong \li{A}{A} \oplus \li{A}{A}$ and let $\phi : \li{A}{A} \rightarrow \li{A}{A} \oplus \li{A}{A}$ be an algebra homomorphism. We set
\[
 (b_0, b_1) := \phi(1)
\]
and notice that for all $a \in A$
\[
 \phi(a) = \phi(a \cdot 1) = a \phi(1) = a(b_0, b_1) = (ab_0, ab_1).
\]
Because $\phi$ is surjective we find $a_0, a_1 \in A$ with
\begin{align*}
 (1,0) &= \phi(a_0) = (a_0 b_0, a_0 b_1) \text{ and } \\
 (0,1) &= \phi(a_1) = (a_1 b_0, a_1 b_1).
\end{align*}
In particular we have $a_0 b_0 = a_1 b_1 = 1$ and $a_0 b_1 = a_1 b_0 = 0$. Because
\[
 \phi(b_0 a_0 + b_1 a _1)
 = (b_0 a_0 b_0 + b_1 a_1 b_0, b_0 a_0 b_1 + b_1 a_1 b_1)
 = (b_0, b_1)
 = \phi(1)
\]
it follows from the injectivity of $\phi$ that $b_0 a_0 + b_1 a_1 = 1$.

Now assume that there exist elements $a_0, a_1, b_0, b_1 \in A$ with $a_0 b_0 = a_1 b_1 = 1$, $a_0 b_1 = a_1 b_0 = 0$ and $b_0 a_0 + b_1 a_1 = 1$. We define
\[
 \psi : \li{A}{A} \rightarrow \li{A}{A} \oplus \li{A}{A}, a \mapsto a(b_0, b_1) = (ab_0, ab_1).
\]
It is clear that $\psi$ is an $A$-module homomorphism. For all $(c_0, c_1) \in \li{A}{A} \oplus \li{A}{A}$ we have
\[
 \psi(c_0 a_0 + c_1 a_1)
 = (c_0 a_0 b_0 + c_1 a_1 b_0, c_0 a_0 b_1 + c_1 a_1 b_1)
 = (c_0, c_1),
\]
so $\psi$ is surjective. For $x \in A$ with $\psi(x) = 0$ we have $(x b_0, x b_1) = (0,0)$, so $(x b_0 a_0, x b_1 a_1) = (0,0)$ and thus
\[
 0 = x b_0 a_0 + x b_1 a_1 = x (b_0 a_0 + b_1 a_1) = x \cdot 1 = x.
\]
So $\psi$ in injective. This shows that $\psi$ is an $A$-module isomorphism and therefore $\li{A}{A} \cong \li{A}{A} \oplus \li{A}{A}$.

One trivial example of such an algebra is $A = 0$.





\section{}
We notice that $\img(h_2 f_1) \subseteq \img g_1$: Because the upper row is exact we have $f_2 f_1 = 0$ and from the commutativity of the diagram it follows that
\[
 0 = h_3 f_2 f_1 =  g_2 h_2 f_1.
\]
Because the lower row is exact this gives us
\[
 \img(h_2 f_1) \subseteq \ker g_2 = \img g_1.
\]
Because $g_1$ is injective it induces an isomorphism $\bar{g}_1 : Y_1 \rightarrow \img g_1$. In particular $1_{Y_1} = \bar{g}_1^{-1} g_1$ and $1_{\img g_1} = g_1 \bar{g}_1^{-1}$. Therefore $h_1 := \bar{g}_1^{-1} h_2 f_1$ is an homomorphism with
\[
 g_1 h_1 = g_1 \bar{g}_1^{-1} h_2 f_1 = 1_{\img g_1} h_2 f_1 = h_2 f_1.
\]
This homomorphism is unique because for $h'_2 : X_1 \rightarrow Y_1$ with $g_1 h'_1 = h_2 f_1$ we have
\[
 h'_1 = 1_{Y_1} h'_1 = \bar{g}_1^{-1} g_1 h'_1 = \bar{g}^{-1}_1 h_2 f_1 = h_1.
\]

















\end{document}
