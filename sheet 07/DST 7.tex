\documentclass[a4paper,10pt]{article}
%\documentclass[a4paper,10pt]{scrartcl}

\usepackage{xltxtra}
\usepackage{amsmath}
\usepackage{amssymb}
\usepackage{amsthm}
\usepackage{mathtools}
\usepackage{paralist}
\usepackage{leftidx}

\theoremstyle{definition}
\newtheorem{own}{}
\newtheorem{lem}[own]{Lemma}
\newtheorem*{claim}{Claim}
\newtheorem{rem}[own]{Remark}
\newtheorem*{ia}{Base case}
\newtheorem*{is}{Induction step}

\newcommand{\N}{\operatorname{\mathbb{N}}}
\newcommand{\Z}{\operatorname{\mathbb{Z}}}
\newcommand{\Q}{\operatorname{\mathbb{Q}}}
\newcommand{\R}{\operatorname{\mathbb{R}}}
\newcommand{\C}{\operatorname{\mathbb{C}}}
\newcommand{\End}{\operatorname{End}}
\newcommand{\Hom}{\operatorname{Hom}}
\newcommand{\img}{\operatorname{img}}
\newcommand{\soc}{\operatorname{soc}}
\newcommand{\rad}{\operatorname{rad}}
\newcommand{\kchar}{\operatorname{char}}
\newcommand{\tr}{\operatorname{tr}}
\newcommand{\id}{\operatorname{id}}
\newcommand{\gen}[1]{\left\langle#1\right\rangle}
\newcommand{\vect}[1]{\begin{pmatrix}#1\end{pmatrix}}
\newcommand{\bvect}[1]{\begin{bmatrix}#1\end{bmatrix}}

\addtocounter{section}{24}

\renewcommand{\thesection}{Exercise \arabic{section}:}
\renewcommand{\thesubsection}{(\roman{subsection})}

\setromanfont[Mapping=tex-text]{Linux Libertine O}
% \setsansfont[Mapping=tex-text]{DejaVu Sans}
% \setmonofont[Mapping=tex-text]{DejaVu Sans Mono}

\parindent 0pt

\title{\textsc{Foundations of representation theory \\ \Large 7. Exercise sheet}}
\author{Jendrik Stelzner}
\date{\today}

\begin{document}
\maketitle




\addtocounter{section}{1}
\section{}
Let $(e_1, \ldots, e_n)$ be a basis of $A$ as a vector space.


\subsection*{(i) $\Rightarrow$ (ii)}
Let $(b_1, \ldots, b_m)$ be a generating set of $M$ as an $A$-module. We can write $x \in M$ as $x = \sum_{j=1}^m a_j b_j$ with $a_j \in A$ for all $j$. We can write each $a_j$ as $a_j = \sum_{i=1}^n \lambda^j_i e_i$ with $\lambda^j_i \in K$ for all $i$. Thus we get
\[
 x = \sum_{j=1}^m a_j b_j =  \sum_{i=1}^n \sum_{j=1}^m \lambda^j_i e_i b_j.
\]
Thus $x$ is a linear combination of the $e_i b_j$. Because $x$ is arbitrary it follows that $\{e_i b_j\}_{i=1,\ldots,n, j=1,\ldots,m}$ is a finite generating set of $M$ as a vector space, so $M$ is finite-dimensional.


\subsection*{(ii) $\Rightarrow$ (i)}
If $(b_1, \ldots, b_m)$ is a basis of $M$ as a vector space, then $(b_1, \ldots, b_m)$ is also a generating set of $M$ as an $A$-module, because $\lambda b_i = \lambda 1_A b_i$ for all $\lambda \in K$ and $i$.


\subsection*{(ii) $\Rightarrow$ (iii)}
This follows directly from $l(M) \leq \dim(V) < \infty$.


\subsection*{$\neg$(ii) $\Rightarrow$ $\neg$(iii)}
We construct an ascending chain $U_0 \subsetneq U_1 \subsetneq U_2 \subsetneq \ldots$ of finite-dimensional submodules of $M$ as follows: We start with $U_0 := 0$. If $U_{n-1}$ is defined we choose some $v \in M \setminus U_{n-1}$ (this is possible because $U_{n-1}$ is finite-dimensional but $M$ is infinite-dimensional). The submodule $W = Av = \gen{e_1 v, \ldots, e_n v}$ of $M$ is not contained in $U_{n-1}$, so $U_{n-1} \subsetneq U_{n-1} + W =: U_n$. $U_n$ is finite-dimensional because $U_{n-1}$ and $W$ are finite-dimensional.

For all $n \in \N$ the filtration
\[
 0 = U_0 \subsetneq U_1 \subsetneq U_2 \subsetneq \ldots \subsetneq U_{n-1} \subsetneq M
\]
is of length $n$, so $l(M) \geq n$ for all $n \in \N$.



\section{}
We can assume that $Q$ is weakly connected: Otherwise, if $Q_1, \ldots, Q_n$ are the weakly connected component of $Q$, it is easy to see that $KQ = \oplus_{i=1}^n KQ_i$ and $C(KQ) = \oplus_{i=1}^n C(KQ_i)$. Also notice that for every quiver $Q$ the center $C(KQ)$ is a subalgebra.

I didn't manage to determine $C(KQ)$ for every weakly connected quiver, but at least for one kind: We look at a quiver with vertices $Q_0 = (1, \ldots, n)$ and arrows $(a_1, \ldots, a_n)$ where $a_i$ goes from $i$ to $i+1$ for $1 \leq i \leq n-1$ and $a_n$ goes from $n$ to $1$. One can see this quiver as a cycle with $n$ vertices.

Let $x \in C(A)$ with $x = \sum_{k=1}^n \lambda_k p_k$ where $\lambda_k \in K$ and $p_k$ is a path in $Q$ for all $k$. We notice that $\lambda_k \neq 0$ implies that $s(p_k) = t(p_k)$: For all $i \in Q_0$ let
\[
 I^s_i := \{1 \leq k \leq n : s(p_k)=i\} \text{ and } I^t_i := \{1 \leq k \leq n : t(p_k)=i\}.
\] Because $x \in C(A)$ we find that
\begin{equation}\label{eq: pfade kreise}
 \sum_{k \in I^t_i} \lambda_k p_k = e_i x = x e_i = \sum_{k \in I^s_i} \lambda_k p_k.
\end{equation}
Because the paths $p_k$ are linear independent we find that $I^s_i = I^t_i$. (This observation holds true for every quiver.)

Because of the special form of $Q$ we can classify all paths $p$ in $Q$ with $s(p) = t(p)$ by $C^n_i$, where $C_i := (a_i, \ldots, a_n, a_1, a_2, \ldots, a_{i-1})$ and $n \in \N$.

So we know that $x = \sum_{k=1}^n \lambda_k C^{\nu_k}_{i_k}$ with $\lambda_k \neq 0$ for all $k$. We find that for every $k$ we have $C_j^{\nu_i}$ as a non-zero linear factor of $x$ for all $i$: This follows directly from the equality given by \eqref{eq: pfade kreise}.
So every $x \in C(A)$ is of the form
\begin{equation*}\label{eq: looks like polynomial}
 x
 = \sum_{k=1}^d \lambda_k \sum_{i=1}^n C_i^k
 = \sum_{k=1}^d \lambda_k \left( \sum_{i=1}^n C_i \right)^k.
\end{equation*}
It is also easy to see, that $C := \sum_{i=1}^n C_i \in C(A)$, because every path $p$ of $Q$ commutes with $C := \sum_{i=1}^n C_i$: If $s(p)=i$ and $t(p)=j$ we get that
\[
 pC = p C^1_i = C^1_j p = Cp.
\]
Because the paths are a $K$-Basis of $KQ$ it follows that $C$ commutes with every element in $KQ$.
Because $C(A)$ as a subalgebra is closed under scalar multiplication, addition and multiplication, we find that
\[
 C(A) = \left\{ \sum_{k=1}^d \lambda_k C^k : \lambda_k \in K \text{ for all } k\right\}.
\]
In particular we find that $C(A) \cong K[T]$.








\section{}
Let
\[
 P := \{p : \text{$p$ is a path from $i$ to $j$ such that there is no path from $j$ to $i$}\}.
\]
First we show that $P \subseteq J(KQ)$: Let $p \in P$ with $s(p) = i$ and $t(p)=j$, i.e. with $p = p e_i = e_j p$ and $e_i q e_j = 0$ for all paths $q$ in $Q$. Let $x \in A$ with $x = \sum_{k=1}^n \lambda_k q_k$ for paths $q_k$ and $\lambda_k \in K$. We find that
\[
 (px)^2
 = \left( \sum_{k=1}^n \lambda_k p q_k \right)^2
 = \sum_{k,l=1}^n \lambda_k \lambda_l p q_k p q_l
 = 0,
\]
because $p q_k p = p e_i q_k e_j p = 0$ for all paths $k$. Because $px$ is nilpotent we find that $1-px$ is invertible. Because $x$ is arbitrary is follows that $p \in J(KQ)$.

Now we show that $P$ is a basis of $J(KQ)$. We already know that $P$ is linear independent, because all paths are linear independent, so all that's left to show is that every $x \in J(KQ)$ can be written as a linear combination of elements in $P$.

Assume that there is some $x \in J(KQ)$ which cannot be written as a linear combination of elements in $P$. Let $x = \sum_{k=1}^n \lambda_k x_k$ be a linear combination of $x$ with $\lambda_k \neq 0$ for all $k$ and the $x_k$ being pairwise different paths. We can assume w.l.o.g. that $x_k \not\in P$ for all $k$, because $P \subseteq J(KQ)$. We can also assume that $s(x_k) = s(x_l)$ and $t(x_k) = t(x_l)$ for all $k,l$, because otherwise we can replace $x$ by $e_{t(x_1)} x e_{s(x_1)} \in e_{t(x_1)} J(KQ) e_{s(x_1)} \subseteq J(KQ)$.

Because $x_1 \not\in P$ we find some path $\bar{x}$ with $s(\bar{x}) = t(x_1)$ and $t(\bar{x}) = s(x_1)$. Because $x \in J(KQ)$ we know that $1-\bar{x}x$ is invertible. Notice that
\[
 y := \bar{x}{x} = \sum_{k=1}^n \lambda_k \underbrace{\bar{x} x_k}_{:= y_k}
\]
is non-zero with $s(y_k) = t(y_k) = s(y_l) = t(y_l) =: i$ for all $k,l$, i.e. the $y_k$ are closed paths from $i$ to $i$.

Because $1-y$ is invertible we find som $y' = \sum_{k=1}^m \mu_k w_k$ with $\mu_k \neq 0$ and the $w_k$ being pairwise different paths, such that $yy' = 1$. So we get
\begin{equation}\label{eq: not so beautiful}
 \sum_{j \in Q_0} e_j
 = 1
 = (1-y) y'
 = y' - yy'
 = \sum_{k=1}^m \mu_k w_k - \sum_{k=1}^n \sum_{l=1}^m \lambda_k \mu_l y_k w_l.
\end{equation}
For $j \in Q_0$ let
\[
 J_j := \{1 \leq k \leq m: t(w_k) = j\}
\]
and $I := J_i$. By multiplying \eqref{eq: not so beautiful} from the left with $e_i$ we get
\[
 e_i = \sum_{k \in I} \mu_k w_k - \sum_{k=1}^n \sum_{l \in I} \lambda_k \mu_l y_k w_l.
\]
which contradicts the linear independence of the paths in $Q$.

















\end{document}
