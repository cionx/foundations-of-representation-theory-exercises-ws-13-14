\documentclass[a4paper,10pt]{article}
%\documentclass[a4paper,10pt]{scrartcl}

\usepackage{xltxtra}
\usepackage{amsmath}
\usepackage{amssymb}
\usepackage{amsthm}
\usepackage{mathtools}
\usepackage{paralist}
\usepackage{leftidx}

\theoremstyle{definition}
\newtheorem{own}{}
\newtheorem{lem}[own]{Lemma}
\newtheorem*{claim}{Claim}
\newtheorem{rem}[own]{Remark}
\newtheorem*{ia}{Base case}
\newtheorem*{is}{Induction step}

\newcommand{\N}{\operatorname{\mathbb{N}}}
\newcommand{\Z}{\operatorname{\mathbb{Z}}}
\newcommand{\Q}{\operatorname{\mathbb{Q}}}
\newcommand{\R}{\operatorname{\mathbb{R}}}
\newcommand{\C}{\operatorname{\mathbb{C}}}
\newcommand{\End}{\operatorname{End}}
\newcommand{\Hom}{\operatorname{Hom}}
\newcommand{\img}{\operatorname{img}}
\newcommand{\soc}{\operatorname{soc}}
\newcommand{\rad}{\operatorname{rad}}
\newcommand{\kchar}{\operatorname{char}}
\newcommand{\tr}{\operatorname{tr}}
\newcommand{\id}{\operatorname{id}}
\newcommand{\gen}[1]{\left\langle#1\right\rangle}
\newcommand{\vect}[1]{\begin{pmatrix}#1\end{pmatrix}}
\newcommand{\bvect}[1]{\begin{bmatrix}#1\end{bmatrix}}

\addtocounter{section}{20}

\renewcommand{\thesection}{Exercise \arabic{section}:}
\renewcommand{\thesubsection}{(\roman{subsection})}

\setromanfont[Mapping=tex-text]{Linux Libertine O}
% \setsansfont[Mapping=tex-text]{DejaVu Sans}
% \setmonofont[Mapping=tex-text]{DejaVu Sans Mono}

\parindent 0pt

\title{\textsc{Foundations of representation theory \\ \Large 7. Exercise sheet}}
\author{Jendrik Stelzner}
\date{\today}

\begin{document}
\maketitle




\addtocounter{section}{1}
\section{}
Let $(e_1, \ldots, e_n)$ be a basis of $A$ as a vector space.


\subsection*{(i) $\Rightarrow$ (ii)}
Let $(b_1, \ldots, b_m)$ be a generating set of $M$ as an $A$-module. We can write $x \in M$ as $x = \sum_{j=1}^m a_j b_j$ with $a_j \in A$ for all $j$. We can write each $a_j$ as $a_j = \sum_{i=1}^n \lambda^j_i e_i$ with $\lambda^j_i \in K$ for all $i$. Thus we get
\[
 x = \sum_{j=1}^m a_j b_j =  \sum_{i=1}^n \sum_{j=1}^m \lambda^j_i e_i b_j.
\]
Thus $x$ is a linear combination of the $e_i b_j$. Because $x$ is arbitrary it follows that $\{e_i b_j\}_{i=1,\ldots,n, j=1,\ldots,m}$ is a finite generating set of $M$ as a vector space, so $M$ is finite-dimensional.


\subsection*{(ii) $\Rightarrow$ (i)}
If $(b_1, \ldots, b_m)$ is a basis of $M$ as a vector space, then $(b_1, \ldots, b_m)$ is also a generating set of $M$ as an $A$-module, because $\lambda b_i = \lambda 1_A b_i$ for all $\lambda \in K$ and $i$.


\subsection*{(ii) $\Rightarrow$ (iii)}
This follows directly from $l(M) \leq \dim(V) < \infty$.


\subsection*{$\neg$(ii) $\Rightarrow$ $\neg$(iii)}
We construct an ascending chain $U_0 \subsetneq U_1 \subsetneq U_2 \subsetneq \ldots$ of finite-dimensional submodules of $M$ as follows: We start with $U_0 := 0$. If $U_{n-1}$ is defined we choose some $v \in M \setminus U_{n-1}$ (this is possible because $U_{n-1}$ is finite-dimensional but $M$ is infinite-dimensional). The submodule $W = Av = \gen{e_1 v, \ldots, e_n v}$ of $M$ is not contained in $U_{n-1}$, so $U_{n-1} \subsetneq U_{n-1} + W =: U_n$. $U_n$ is finite-dimensional because $U_{n-1}$ and $W$ are finite-dimensional.

For all $n \in \N$ the filtration
\[
 0 = U_0 \subsetneq U_1 \subsetneq U_2 \subsetneq \ldots \subsetneq U_{n-1} \subsetneq M
\]
is of length $n$, so $l(M) \geq n$ for all $n \in \N$.








\end{document}
