\documentclass[a4paper,10pt]{article}
%\documentclass[a4paper,10pt]{scrartcl}

\usepackage{xltxtra}
\usepackage{amsmath}
\usepackage{amssymb}
\usepackage{amsthm}
\usepackage{mathtools}
\usepackage{paralist}
\usepackage{leftidx}

\theoremstyle{definition}
\newtheorem{own}{}
\newtheorem{lem}[own]{Lemma}
\newtheorem*{claim}{Claim}
\newtheorem{rem}[own]{Remark}
\newtheorem*{ia}{Base case}
\newtheorem*{is}{Induction step}

\newcommand{\N}{\operatorname{\mathbb{N}}}
\newcommand{\Z}{\operatorname{\mathbb{Z}}}
\newcommand{\Q}{\operatorname{\mathbb{Q}}}
\newcommand{\R}{\operatorname{\mathbb{R}}}
\newcommand{\C}{\operatorname{\mathbb{C}}}
\newcommand{\End}{\operatorname{End}}
\newcommand{\Hom}{\operatorname{Hom}}
\newcommand{\img}{\operatorname{img}}
\newcommand{\soc}{\operatorname{soc}}
\newcommand{\rad}{\operatorname{rad}}
\newcommand{\kchar}{\operatorname{char}}
\newcommand{\tr}{\operatorname{tr}}
\newcommand{\id}{\operatorname{id}}
\newcommand{\gen}[1]{\left\langle#1\right\rangle}
\newcommand{\vect}[1]{\begin{pmatrix}#1\end{pmatrix}}
\newcommand{\bvect}[1]{\begin{bmatrix}#1\end{bmatrix}}

\addtocounter{section}{28}

\renewcommand{\thesection}{Exercise \arabic{section}:}
\renewcommand{\thesubsection}{(\roman{subsection})}

\setromanfont[Mapping=tex-text]{Linux Libertine O}
% \setsansfont[Mapping=tex-text]{DejaVu Sans}
% \setmonofont[Mapping=tex-text]{DejaVu Sans Mono}

\parindent 0pt

\title{\textsc{Foundations of representation theory \\ \Large 8. Exercise sheet}}
\author{Jendrik Stelzner}
\date{\today}

\begin{document}
\maketitle





\section{}
We will assume that the vertices of $Q$ are ordered in the most obvious way. We define the subalgebra $B$ of $M_n(K)$ as
\[
 B := \{M = (m_{ij})_{ij} \in M_n(K) : m_{ij} = 0 \text{ for all } j > i \}.
\]
We will show that $KQ \cong B \cong A$.

For all $1 \leq i \leq j \leq n$  let $p_{ij}$ be the unique path in $Q$ from $i$ to $j$ and for all $1 \leq i, j \leq n$ let $E_{ij} \in M_n(K)$ be the matrix with $1$ as the $(i,j)$-entry and $0$ otherwise. ($E_{ij}$ maps $e_j$ to $e_i$.) We know that $(p_{ij})_{1 \leq i \leq j \leq n}$ is a basis of $KQ$, $(E_{ij})_{1 \leq j \leq i \leq n}$ is a basis of $B$ and $(E_{ij})_{1 \leq i \leq j \leq n}$ is a basis of $A$.

Let $\phi : KQ \rightarrow B$ be the linear map given by $\phi(p_{ij}) = E_{ji}$ for all $1 \leq i \leq j \leq n$. $\phi$ is a $K$-algebra homomorphism since for all $1 \leq i \leq j \leq n$ and $1 \leq l \leq k \leq n$
\begin{equation*}
 \phi(p_{ij} p_{lk})
 = \phi(\delta_{ik} p_{lj})
 = \delta_{ik} \phi(p_{lj})
 = \delta_{ik} E_{jl}
 = E_{ji} E_{kl}
 = \phi(p_{ij}) \phi(p_{lk}).
\end{equation*}
$\phi$ is an isomorphism, because for the linear map $\psi: B \rightarrow KQ$ given by $\psi(E_{ij}) = p_{ji}$ for all $1 \leq i \leq j \leq n$ we have $\phi \psi = \id_B$ and $\psi \phi = \id_{KQ}$. Thus we have $KQ \cong B$.

To show that $B \cong A$ we notice that for the matrix
\[
 S :=
 \begin{pmatrix}
    &         & 1 \\
    & \diagup &   \\
  1 &         & 
 \end{pmatrix}
 \in M_n(K)
\]
with $S^2 = 1$ the map
\[
 f : M_n(K) \rightarrow M_n(K), F \mapsto SFS
\]
is an vector space automorphism with $f^2 = \id$. $f$ is an algebra isomorphism because for all $F,G \in M_n(K)$
\[
 f(FG) = SFGS = SFS^2GS = f(F)f(G).
\]
We also notice that $f$ maps the Basis $(E_{ij})_{1 \leq i \leq j \leq n}$ of $A$ to the basis $(E_{ij})_{1 \leq j \leq i \leq n}$ of $B$, thus $f_{|A} \rightarrow f_{|B}$ is an algebra isomorphism.





\section{}
We name the vertices of $Q$ as $1$ and $2$ with $s(\alpha) = t(\alpha) = 1$ and the arrow from $1$ to $2$ as $p$. By definition
\[
 P := \{e_1, e_2, p\} \cup \bigcup_{n \geq 1} \{\alpha^n, p\alpha^n \}
\]
is a basis of $KQ$. It is obvious that
\[
 B =
 \begin{pmatrix}
  K[T] & 0 \\
  K[T] & K
 \end{pmatrix}
\]
is a $K$-algebra via the usual matrix multiplication. We define the linear map $\phi : KQ \rightarrow B$ by
\begin{gather*}
 \phi(e_1) = \vect{1 & 0 \\ 0 & 0},
 \phi(e_2) = \vect{0 & 0 \\ 0 & 1},
 \phi(p) = \vect{0 & 0 \\ 1 & 0} \text{ and} \\
 \phi(\alpha^n) = \vect{T^n & 0 \\ 0 & 0},
 \phi(p\alpha^n) = \vect{0 & 0 \\ T^n & 0} \text{ for all } n \geq 1.
\end{gather*}
It is clear that $\phi$ induces a bijection between $P$ and a basis of $B$, so $\phi$ is a vector space isomorphism. It is also easy to see that $\phi$ is an algebra homomorphism, because $\phi(xy) = \phi(x)\phi(y)$ for all $x,y \in P$ (this can be directly shown by some boring matrix multiplication which I will not include here). Thus $KQ \cong B$.

The ideal $I = (\alpha^2)$ in $KQ$ generated by the path $\alpha^2$ corresponds to the ideal $J = (\phi(\alpha)^2)$ in $B$ generated by $\phi(\alpha)^2$. Because
\[
 \begin{pmatrix}
  a & 0 \\
  b & c
 \end{pmatrix}
 \underbrace{\begin{pmatrix}
  T^2 & 0 \\
  0 & 0
 \end{pmatrix}}_{= \phi(\alpha)^2}
 \begin{pmatrix}
  d & 0 \\
  e & f
 \end{pmatrix}
 =
 \begin{pmatrix}
  adT^2 & 0 \\
  bdT^2 & 0
 \end{pmatrix}
\]
we find that
\[
 J
 =
 (\phi(\alpha)^2)
 =
 \begin{pmatrix}
  (T^2) & 0 \\
  (T^2) & 0
 \end{pmatrix}.
\]
Thus $\phi$ induces an algebra isomorphism $\bar{\phi}$ between the algebras $KQ/I$ and $B/J = A$.









\end{document}
