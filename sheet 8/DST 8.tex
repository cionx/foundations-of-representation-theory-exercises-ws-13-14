\documentclass[a4paper,10pt]{article}
%\documentclass[a4paper,10pt]{scrartcl}

\usepackage{xltxtra}
\usepackage{amsmath}
\usepackage{amssymb}
\usepackage{amsthm}
\usepackage{mathtools}
\usepackage{paralist}
\usepackage{leftidx}

\theoremstyle{definition}
\newtheorem{own}{}
\newtheorem{lem}[own]{Lemma}
\newtheorem*{claim}{Claim}
\newtheorem{rem}[own]{Remark}
\newtheorem*{ia}{Base case}
\newtheorem*{is}{Induction step}

\newcommand{\N}{\operatorname{\mathbb{N}}}
\newcommand{\Z}{\operatorname{\mathbb{Z}}}
\newcommand{\Q}{\operatorname{\mathbb{Q}}}
\newcommand{\R}{\operatorname{\mathbb{R}}}
\newcommand{\C}{\operatorname{\mathbb{C}}}
\newcommand{\End}{\operatorname{End}}
\newcommand{\Hom}{\operatorname{Hom}}
\newcommand{\img}{\operatorname{img}}
\newcommand{\soc}{\operatorname{soc}}
\newcommand{\rad}{\operatorname{rad}}
\newcommand{\kchar}{\operatorname{char}}
\newcommand{\tr}{\operatorname{tr}}
\newcommand{\id}{\operatorname{id}}
\newcommand{\gen}[1]{\left\langle#1\right\rangle}
\newcommand{\vect}[1]{\begin{pmatrix}#1\end{pmatrix}}
\newcommand{\bvect}[1]{\begin{bmatrix}#1\end{bmatrix}}

\addtocounter{section}{28}

\renewcommand{\thesection}{Exercise \arabic{section}:}
\renewcommand{\thesubsection}{(\roman{subsection})}

\setromanfont[Mapping=tex-text]{Linux Libertine O}
% \setsansfont[Mapping=tex-text]{DejaVu Sans}
% \setmonofont[Mapping=tex-text]{DejaVu Sans Mono}

\parindent 0pt

\title{\textsc{Foundations of representation theory \\ \Large 8. Exercise sheet}}
\author{Jendrik Stelzner}
\date{\today}

\begin{document}
\maketitle





\section{}
We will assume that the vertices of $Q$ are ordered in the most obvious way. For all $1 \leq i \leq j \leq n$  let $p_{ij}$ be the unique path in $Q$ from $i$ to $j$ and for all $1 \leq i, j \leq n$ let $E_{ij} \in M_n(K)$ be the matrix with $1$ as the $(i,j)$-entry and $0$ otherwise. ($E_{ij}$ maps $e_j$ to $e_i$.) We know that $(p_{ij})_{1 \leq i \leq j \leq n}$ is a basis of $KQ$ and $(E_{ij})_{1 \leq i \leq j \leq n}$ is a basis of $A$. Let $\phi : KQ \rightarrow A$ be the linear map given by $\phi(p_{ij}) = E_{n-j,n-i}$ for all $1 \leq i \leq j \leq n$. $\phi$ is a $K$-algebra homomorphism since for all $1 \leq i \leq j \leq n$ and $1 \leq l \leq k \leq n$
\begin{align*}
 \phi(p_{ij} p_{lk})
 &= \phi(\delta_{ki} p_{lj})
 = \delta_{ki} \phi(\phi_{lj})
 = \delta_{ki} E_{n-j, n-l} \\
 &= E_{n-j, n-i} E_{n-k, n-l}
 = \phi(p_{ij}) \phi(p_{lk}).
\end{align*}
$\phi$ is an isomorphism, because for the linear map $\psi: A \rightarrow KQ$ given by $\psi(E_{ij}) = p_{n-j,n-i}$ for all $1 \leq i \leq j \leq n$ we have $\phi \psi = \id_A$ and $\psi \phi = \id_{KQ}$.








\end{document}
