\documentclass[a4paper,10pt]{article}
%\documentclass[a4paper,10pt]{scrartcl}

\usepackage{xltxtra}
\usepackage{amsmath}
\usepackage{amssymb}
\usepackage{amsthm}
\usepackage{mathtools}
\usepackage{enumerate}

\newcommand{\End}{\operatorname{End}}
\newcommand{\Hom}{\operatorname{Hom}}
\newcommand{\kchar}{\operatorname{char}}
\newcommand{\tr}{\operatorname{tr}}
\newcommand{\gen}[1]{\left\langle#1\right\rangle}
\newcommand{\vect}[1]{\begin{pmatrix}#1\end{pmatrix}}

\addtocounter{section}{8}
\renewcommand{\thesection}{Exercise \arabic{section}:}

\setromanfont[Mapping=tex-text]{Linux Libertine O}
% \setsansfont[Mapping=tex-text]{DejaVu Sans}
% \setmonofont[Mapping=tex-text]{DejaVu Sans Mono}

\parindent 0pt

\title{Foundations of representation theory \\ 3. exercise sheet}
\author{Jendik Stelzner}
\date{\today}

\begin{document}
\maketitle





\section{}





\section{}





\section{}
Let $e_1, \ldots, e_n$ be the canonical basis of $K^n$ and $\phi, \psi \in \Hom_{K}(K^n)$ be defined as
\[
 \phi(e_i) :=
 \begin{cases}
  0       & \text{ if } i=1 \\
  e_{i-1} & \text{ otherwise}
 \end{cases}
 \text{ and }
 \psi(e_i) :=
 \begin{cases}
  e_1     & \text{ if } i=n \\
  e_{i+1} & \text{ otherwise}
 \end{cases}.
\]
Let $U \subseteq K^n$ be a submodule. If $U \neq 0$ we find $v \in U$ with $v \neq 0$. Since $e_1, \ldots, e_n$ is a basis of $K^n$ we find $\lambda_1, \ldots, \lambda_n \in K$ with $v = \sum_{i=1}^n \lambda_i e_i$. Let $m := \max \{i \in \{1, \ldots, n\} : \lambda_i \neq 0\}$; this is well-defined because $v \neq 0$, so $\lambda_i \neq 0$ for some $i \in \{1, \ldots, n\}$. Because $U$ is a submodule we find that $e_1 = \lambda_m^{-1} \phi^m(v) \in U$. So for all $i \in \{1, \ldots, n\}$ we get $e_i = \psi^i(e_1) \in U$. Since $\{e_1, \ldots, e_n\} \subseteq U$ it follows that $U = K^n$. So every submodule $(K^n, \phi, \psi)$ is either $0$ or $K^n$, which means that $(K^n,\phi,\psi)$ is an $n$-dimensional simple $2$-module.






\section{}
Let $K$ be a field with $\kchar(K) = 0$ and $(V,\phi_1,\phi_2)$ is a  $2$-module such that $V \neq 0$ and $[\phi_1, \phi_2] = 1$. Assume that $V$ is finite-dimensional. Because $V \neq 0$ we know that $n := \dim V \geq 1$. Let $v_1, \ldots, v_n$ be a Basis von $V$ and $\Phi_1$ and $\Phi_2$ the coordinate matrix of $\phi_1, \phi_2$ with respect to the basis $v_1, \ldots, v_n$ respectively. Because $[\phi_1, \phi_2] = 1$ we find that $\Phi_1 \Phi_2 - \Phi_2 \Phi_1 = I_n$. It follows, that
\[
 0
 = \tr \Phi_1 \tr \Phi_2 - \tr \Phi_2 \tr \Phi_1
 = \tr (\Phi_1\Phi_2-\Phi_2\Phi_1)
 = \tr I_n
 = n \cdot 1
 \neq 0.
\]
This shows that $(V,\phi_1,\phi_2)$ must be finite dimensional. (We know that such a module exists, because $\left(K[T], T\cdot, \frac{d}{dT}\right)$ is one.)








\end{document}
