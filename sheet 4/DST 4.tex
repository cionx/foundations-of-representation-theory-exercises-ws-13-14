\documentclass[a4paper,10pt]{article}
%\documentclass[a4paper,10pt]{scrartcl}

\usepackage{xltxtra}
\usepackage{amsmath}
\usepackage{amssymb}
\usepackage{amsthm}
\usepackage{mathtools}
\usepackage{paralist}

\theoremstyle{definition}
\newtheorem{own}{}
\newtheorem{lem}[own]{Lemma}
\newtheorem*{claim}{Claim}
\newtheorem{rem}[own]{Remark}
\newtheorem*{ia}{Base case}
\newtheorem*{is}{Induction step}

\newcommand{\N}{\operatorname{\mathbb{N}}}
\newcommand{\Z}{\operatorname{\mathbb{Z}}}
\newcommand{\Q}{\operatorname{\mathbb{Q}}}
\newcommand{\R}{\operatorname{\mathbb{R}}}
\newcommand{\C}{\operatorname{\mathbb{C}}}
\newcommand{\End}{\operatorname{End}}
\newcommand{\Hom}{\operatorname{Hom}}
\newcommand{\img}{\operatorname{im}}
\newcommand{\soc}{\operatorname{soc}}
\newcommand{\rad}{\operatorname{rad}}
\newcommand{\kchar}{\operatorname{char}}
\newcommand{\tr}{\operatorname{tr}}
\newcommand{\gen}[1]{\left\langle#1\right\rangle}
\newcommand{\vect}[1]{\begin{pmatrix}#1\end{pmatrix}}

\addtocounter{section}{12}
\renewcommand{\thesection}{Exercise \arabic{section}:}
\renewcommand{\thesubsection}{(\roman{subsection})}

\setromanfont[Mapping=tex-text]{Linux Libertine O}
% \setsansfont[Mapping=tex-text]{DejaVu Sans}
% \setmonofont[Mapping=tex-text]{DejaVu Sans Mono}

\parindent 0pt

\title{\textsc{Foundations of representation theory \\ \Large 4. Exercise sheet}}
\author{Jendrik Stelzner}
\date{\today}

\begin{document}
\maketitle





\section{}
A submodule $S \subseteq N(\lambda)$ is simple if and only if it is of the form
\[
 S = \gen{\sum_{i=1}^t \mu_i e_{i 1}}
\]
with $\mu_1, \ldots, \mu_t \in K$ and $\mu_j \neq 0$ for at least one $j \in \{1,\ldots,n\}$:
Since
\[
 \phi\left(\sum_{i=1}^t \mu_i e_{i 1}\right) = \sum_{i=1}^t \mu_i \phi(e_{i 1}) = 0
\]
and $\sum_{i=1}^t \mu_i e_{i 1} \neq 0$ these subspaces are $1$-dimensional submodules, which are always simple. If $S \subseteq N(\lambda)$ is simple then $S$ is nonzero, and because $\phi$ is nilpotent there is $v \in S\setminus\{0\}$ with $\phi(v) = 0$. So the subspace $\gen{v} \subseteq S$ is a nonzero submodule of $S$, so $S = \gen{v}$.
Thus we get
\[
 \soc(N(\lambda))
 = \sum_{\substack{S \subseteq N(\lambda) \\ S \text{ semisimple}}} S
 = \sum_{\substack{S \subseteq N(\lambda) \\ S \text{ simple}}} S
 = \gen{e_{1 1}, e_{2 1}, \ldots, e_{t 1}}.
\]

To determine $\rad(N(\lambda))$ we notice that for every $s \in \{1, \ldots, t\}$ the subspace
\[
 M_s := \gen{ \left(\bigcup_{i=1}^t \bigcup_{j=1}^{\lambda_i} \{e_{ij}\}\right) \setminus \{e_{s \lambda_s}\} }
\]
is a maximal submodule of $N(\lambda)$:

It is clear that $M_s$ is a submodule. To show that $M_s$ is maximal in $N(\lambda)$ we notice that for a submodule $M$ of $N(\lambda)$ with $M_s \subsetneq M \subseteq N(\lambda)$ there exists $v \in M$ and $\mu_{ij} \in K$ for $i=1,\ldots,t$, $j=1,\ldots,\lambda_i$ with $\mu_{s \lambda_s} = 1$ such that
\[
 v = \sum_{i=1}^t \sum_{j=1}^{\lambda_i} \mu_{ij} e_{ij},
\]
because otherwise $v \in M_s$ for all $v \in M$ and thus $M \subseteq M_s$. Together with $M_s \subsetneq M$ this implies that $e_{ij} \in M$ for $i=1,\ldots,t$ and $j=1,\ldots,\lambda_i$. So $M = N(\lambda)$.

Because $M_s$ is maximal for $s = 1,\ldots,t$ we find that
\[
 \rad(N(\lambda))
 \subseteq \bigcap_{s=1}^t M_s
 = \gen{ \bigcup_{i=1}^t \bigcup_{j=1}^{\lambda_i-1} \{e_{ij}\} } =: r_0.
\]
We also find that $r_0 \supseteq \rad(N(\lambda))$: If $W \subsetneq N(\lambda)$ is a maximal submodule with $r_0 \nsubseteq W$ then there is some $s \in \{1,\ldots,t\}$ with $e_{s (\lambda_s-2)} \not\in W$, which also means that $e_{s (\lambda_s-1)} \not\in W$, because
\[
 \phi(e_{s (\lambda_s-1)}) = e_{s (\lambda_s-2)}.
\]

Because $W$ is maximal in $N(\lambda)$ the factor module $N(\lambda)/W$ is simple, and because $N(\lambda)/W \cong N(\lambda')$ for some $\lambda' = (\lambda'_1, \ldots, \lambda'_t)$ with $\lambda'_i \leq \lambda_i$ for $i=1,\ldots,t$ this means that $N(\lambda)/W$ is one-dimensional. So $W$ has dimension
\begin{equation}\label{eq:wdim}
 \dim N(\lambda)-1 = \left(\sum_{i=1}^t \lambda_i\right)-1.
\end{equation}

Because $e_{s(\lambda_s-2)} \not \in W$ we find that $\gen{e_{s(\lambda_s-2)}} \cap W = 0$ and that $W$ is a proper subspace of $W + \gen{e_{s(\lambda_s-2)}}$. Therefore \eqref{eq:wdim} implies that $W + \gen{e_{s(\lambda_s-2)}} = N(\lambda)$. So $N(\lambda) = W \oplus \gen{e_{s(\lambda_s-2)}}$ (as vector spaces). So there are $\mu \in K$ and $v \in W$, $v = \sum_{i=1}^t \sum_{j=1}^{\lambda_i} \mu_{ij} e_{ij}$ with $\mu_{ij} \in K$ and $\mu_{s (\lambda-2)} = 0$, such that
\[
 e_{s (\lambda-1)} = \mu e_{s (\lambda-2)} + v = \mu e_{s (\lambda-2)} + \sum_{i=1}^t \sum_{j=1}^{\lambda_i} \mu_{ij} e_{ij}.
\]
Because the $e_{ij}$ are linear independent we get $\mu = 0$, so $e_{s (\lambda-1)} = v \in W$, which is a contradiction to $e_{s (\lambda-1)} \not \in W$. So for every maximal submodule $W \subsetneq N(\lambda)$ we have $r_0 \subseteq W$, so $r_0 \subseteq \rad(N(\lambda))$.





\section{}


\begin{lem}\label{lem:cool}
 Let $V$ and $W$ be modules and $f : V \rightarrow W$ a module homomorphism. Then the following hold:
 \begin{enumerate}[i)]
  \item For every submodule $U \subseteq W$ the preimage $f^{-1}(U)$ is a submodule of $V$. \label{lem:urbild}
  \item If $U \subseteq V$ is a submodule, $\pi : V \twoheadrightarrow V\!/U$ the canonical projection and $W \subseteq V\!/U$ a maximal submodule, then the preimage $\pi^{-1}(W)$ is a maximal submodule of $V$.\label{lem:urbildmax}
  \item If $W \subseteq V$ is maximal in $V$ and $U \subseteq W$ is a submodule then $W\!/U$ is maximal in $V\!/U$ and $\pi^{-1}(W\!/U) = W$. \label{lem:bildmax}
 \end{enumerate}
\end{lem}
\begin{proof}
 \subsubsection*{\ref{lem:urbild})}
  We know from algebra that $f^{-1}(U)$ is an additive subgroup of $V$. $f^{-1}(U)$ is also a subspace of $V$, since for all $\lambda \in K$
  \[
   v \in f^{-1}(U)
   \Rightarrow f(v) \in U
   \Rightarrow f(\lambda v) = \lambda f(v) \in U
   \Rightarrow \lambda v \in f^{-1}(U).
  \]
  $f^{-1}(U)$ is a submodule of $V$, since $f$ is a module homomorphism and $U$ a submodule and so for all $j \in J$ and $v \in f^{-1}(U)$
  \[
   f( \phi_j(v)) = \psi_j( f(v) ) \in U,
  \]
  so $\phi_j(v) \in f^{-1}(U)$.
 \subsubsection*{\ref{lem:urbildmax})}
  From \ref{lem:urbild}) it follows that $f^{-1}(W)$ is a submodule of $V$. Let $W' \subseteq V$ be a arbitrary submodule with
  \begin{equation}\label{eq:urbildmax}
   \pi^{-1}(W) \subseteq W' \subseteq V.
  \end{equation}
  Then $W \subseteq W'\!/U \subseteq V\!/U$, so the maximality of $W$ implies that $W'\!/U = W$ or $W'\!/U = V\!/U$. If $W'\!/U = W$ then
  \[
   W' \subseteq \pi^{-1}(W'\!/U) = \pi^{-1}(W),
  \]
  and with \eqref{eq:urbildmax} we get $W' = \pi^{-1}(W)$. If $W'\!/U = V\!/U$ then the the isomorphism theorems imply that
  \[
   V\!/W' \cong (V\!/U)/(W'\!/U) = 0,
  \]
  so $W' = V$.
 \subsubsection*{\ref{lem:bildmax})}
  $W\!/U$ is a proper submodule of $V\!/U$, because $W$ is maximal in $V$ and so
  \[
   (V\!/U)/(W\!/U) \cong \underbrace{V\!/W}_{\text{simple}} \neq 0.
  \]
  Let $W' \subseteq V\!/U$ be a submodule with
  \[
   W\!/U \subseteq W' \subsetneq V\!/U.
  \]
  We know from \ref{lem:urbild}) that $\pi^{-1}(W')$ is a submodule of $V$, and obviously
  \[
    W \subseteq \pi^{-1}(W\!/U) \subseteq \pi^{-1}(W') \subsetneq V.
  \]
  So the maximality of $W$ implies that $W = \pi^{-1}(W')$. So
  \[
   W\!/U = \pi(W) = \pi(\pi^{-1}(W')) = W'.
  \]
  This shows that $W\!/U$ is maximal in $V\!/U$. Now \textbf{\ref{lem:cool} \ref{lem:urbildmax})} implies that $\pi^{-1}(W\!/U)$ is maximal is $V$. Because $W \subseteq \pi^{-1}(W\!/U)$ it follows that $W = \pi^{-1}(W\!/U)$.
\end{proof}


\subsection*{(ii)}
If $V\!/U$ does not have any maximal submodules we get $\rad(V\!/U) = V\!/U$ and
\[
 (U+\rad(V))/U \subseteq V\!/U = \rad(V\!/U)
\]
If $V\!/U$ does have maximal submodules, then by definition we get
\begin{align*}
 &\; (U + \rad(V))/U \subseteq \rad(V\!/U) \\
 \Leftrightarrow&\; (U + \rad(V))/U \subseteq \bigcap_{\substack{W \subseteq V\!/U \\ W \text{ maximal}}} W \\
 \Leftrightarrow&\; (U + \rad(V))/U \subseteq W \text{ for all } W \subseteq V\!/U \text{ maximal}.
\end{align*}
Let $W \subseteq V\!/U$ be an arbitrary maximal submodule. Using Lemma \textbf{\ref{lem:cool} \ref{lem:urbildmax})} we find that $\pi^{-1}(W)$ is a maximal submodule of $V$, where $\pi : V \twoheadrightarrow V/U$ is the canonical projection. Obviously we have $U \subseteq \pi^{-1}(W)$, and because $\pi^{-1}(W)$ is maximal in $V$ we also get $\rad(V) \subseteq \pi^{-1}(W)$, so
\[
 U+\rad(V) \subseteq \pi^{-1}(W),
\]
and thus
\[
 (U+\rad(V))/U = \pi(U+\rad(V)) \subseteq \pi(\pi^{-1}(W)) = W.
\]


\subsection*{(i)}
From part \textbf{(ii)} we get
\[
 \rad(V)/U = (U+\rad(V))/U \subseteq \rad(V\!/U),
\]
so all that’s left to show is $\rad(V\!/U) \subseteq \rad(V)/U$. If $V$ has no maximal submodules, then $\rad(V) = V$ and
\[
 \rad(V\!/U) \subseteq V\!/U = \rad(V)/U.
\]
If $V$ has maximal submodules, then let $W$ be an arbitrary maximal submodule of $V$. Since $U \subseteq \rad(V) \subseteq W$ we know from \textbf{\ref{lem:cool} \ref{lem:bildmax})} that $W\!/U$ is maximal in $V\!/U$ and $\pi^{-1}(W\!/U) = W$. So $\rad(V\!/U) \subseteq W\!/U$ and thus
\[
 \pi^{-1}(\rad(V\!/U)) \subseteq \pi^{-1}(W\!/U) = W.
\]
Because $W$ is arbitrary it follows that
\[
 \pi^{-1}(\rad(V\!/U)) \subseteq \rad(V)
\]
and from this it follows that
\[
 \rad(V\!/U) = \pi(\pi^{-1}(\rad(V\!/U))) \subseteq \pi(\rad(V)) = \rad(V)/U.
\]





\section{}

\subsection*{$\neg (ii) \Rightarrow \neg (i)$}
Assume that $V = U \oplus C$ is a direct sum decomposition with $U$ simple.
\begin{claim}
 $C$ is maximal in V.
\end{claim}
With this we find that
\[
 U \subseteq \soc(V) \text{ and } \rad(V) \subseteq C
\]
because $U$ is simple and $C$ is maximal in $V$. Because $U$ nonzero with $U \cap C = 0$, this implies that $\soc(V) \nsubseteq \rad(V)$.
\begin{proof}[Proof of the claim]
 Let $C' \subseteq V$ be a submodule with $C \subseteq C' \subseteq V$. Let $C'' := C'\!\cap U$. Because $U$ is simple we know that $C'' = 0$ or $C'' = U$. If $C'' = 0$ then by using modularity and $C \subseteq C'$ we find
 \[
  C = C + C'' = C + (U \cap C') = (C + U) \cap C' = V \cap C' = C'.
 \]
 If $C'' = U$ we get
 \[
  U = C'' = C' \cap U, \text{ so } V = U \oplus C \subseteq U + C' = C', \text{ so } C' = V. \qedhere
 \]
\end{proof}




\subsection*{$(ii) \Rightarrow (i)$}
Assume that $V$ does not have a simple direct summand. If $V$ has no maximal submodule, then $\soc(V) \subseteq \rad(V) = V$.

If $V$ does have at least one maximal submodule then by definition
\[
 \soc(V) \subseteq \rad(V) \Leftrightarrow S \subseteq U \text{ for all } S \subseteq V \text{ simple and all } U \subseteq V \text{ maximal}.
\]
Assume that $S \subseteq V$ is simple and $U \subseteq W$ is maximal with $S \nsubseteq U$. Then $S \cap U \neq S$ and $S \subsetneq S+U$. Because $S$ is simple this implies $S \cap U = 0$, and because $U$ is maximal it implies $S+U = V$. So $V = S \oplus U$. This is a contradiction to the assumption that $V$ does not have a simple direct summand. So $S \subseteq U$ for all for all $S \subseteq V$ simple and all $U \subseteq V$ maximal.





\section{}


\subsection{}
$0 \subseteq K[T]$ is the only small submodule of $(K[T],T\cdot)$: Let $0 \subsetneq U \subseteq K[T]$ be a small submodule. We know that $U = (a)$ for some $a \in K[T]$; because $U$ has to be nonzero and proper we know that $\deg a \geq 1$. We find an irreducible polynomial $p \in K[T]$ with $p \nmid a$. So $(a) + (p) = (1) = K[T]$. Because $(p)$ proper submodule of $(K[T],T\cdot)$ this shows that $U$ is not small.

In $N(\infty)$ all nonzero submodules are small: For all nonzero submodules $V, V' \subseteq N(\infty)$ we have $N(1) \subseteq V, V'$ as a nonzero submodule, so $V \cap V' \supseteq N(1)$ is nonzero.

\subsection{}
Both modules are uniform:

Let $U, U' \subseteq K[T]$ be nonzero submodules. We know that $U = (a)$ and $U' = (b)$ for $a,b \in K[T] \setminus \{0\}$. Thus $(ab)$ is a nonzero submodule of both $U$ and $U'$ (because $K[T]$ has no zero divisors), so $U \cap U' \supseteq (ab)$ is nonzero.

Let $V, V' \subseteq N(\infty)$ be nonzero submodules. We know that $V = N(i)$ and $V' = N(j)$ for $i,j \in \N \setminus \{0\}$. So $V \cap V' \supseteq N(1)$ is nonzero.

\subsection{}
Let $V = N(\infty)$ and $U := N(1) \subseteq V$. $U$ is large in $V$: For every nonzero submodule $U' \subseteq V$ we have $N(1) \subseteq U'$, so $U \cap U' \supseteq N(1)$ is nonzero. $U$ is also small: For every proper submodule $U'' \subseteq V$ we have $U'' = N(i)$ for some $i \in \N$. So $U + U'' = N(1) + N(i) = N(\max\{1,i\})$ is a proper submodule of $V$.





\end{document}
