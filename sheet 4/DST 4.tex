\documentclass[a4paper,10pt]{article}
%\documentclass[a4paper,10pt]{scrartcl}

\usepackage{xltxtra}
\usepackage{amsmath}
\usepackage{amssymb}
\usepackage{amsthm}
\usepackage{mathtools}
\usepackage{paralist}

\theoremstyle{definition}
\newtheorem{own}{}
\newtheorem{lem}[own]{Lemma}
\newtheorem*{claim}{Claim}
\newtheorem{rem}[own]{Remark}
\newtheorem*{ia}{Base case}
\newtheorem*{is}{Induction step}

\newcommand{\End}{\operatorname{End}}
\newcommand{\Hom}{\operatorname{Hom}}
\newcommand{\img}{\operatorname{im}}
\newcommand{\soc}{\operatorname{soc}}
\newcommand{\rad}{\operatorname{rad}}
\newcommand{\kchar}{\operatorname{char}}
\newcommand{\tr}{\operatorname{tr}}
\newcommand{\gen}[1]{\left\langle#1\right\rangle}
\newcommand{\vect}[1]{\begin{pmatrix}#1\end{pmatrix}}

\addtocounter{section}{12}
\renewcommand{\thesection}{Exercise \arabic{section}:}

\setromanfont[Mapping=tex-text]{Linux Libertine O}
% \setsansfont[Mapping=tex-text]{DejaVu Sans}
% \setmonofont[Mapping=tex-text]{DejaVu Sans Mono}

\parindent 0pt

\title{\textsc{Foundations of representation theory \\ \Large 4. Exercise sheet}}
\author{Jendrik Stelzner}
\date{\today}

\begin{document}
\maketitle





\section{}





\section{}





\section{}

\subsection*{$\neg (ii) \Rightarrow \neg (i)$}
Assume that $V = U \oplus C$ is a direct sum decomposition with $U$ simple.
\begin{claim}
 $C$ is maximal in V.
\end{claim}
With this we find that
\[
 U \subseteq \soc(V) \text{ and } \rad(V) \subseteq C
\]
because $U$ is simple and $C$ is maximal in $V$. Because $U$ nonzero with $U \cap C = 0$, this implies that $\soc(V) \nsubseteq \rad(V)$.
\begin{proof}[Proof of the claim]
 Let $C' \subseteq V$ be a submodule with $C \subseteq C' \subseteq V$. Let $C'' := C' \cap U$. Because $V$ is simple we know that $C'' = 0$ or $C'' = U$. If $C'' = 0$ then
 \[
  C = C + C'' = C + (U \cap C') = (C + U) \cap C' = V \cap C' = C'.
 \]
 If $C'' = U$ we get
 \[
  U = C'' = C' \cap U, \text{ so } V = U \cap C \subseteq C', \text{ so } C' = V. \qedhere
 \]
\end{proof}




\subsection*{$(ii) \Rightarrow (i)$}
Assume that $V$ does not have a simple direct summand. If $V$ has no maximal submodule, then $\soc(V) \subseteq \rad(V) = V$ is trivial.
If $V$ does have at least one maximal submodule it is easy to see that
\[
 \soc(V) \subseteq \rad(V) \Leftrightarrow S \subseteq U \text{ for all } S \subseteq V \text{ simple and all } U \subseteq V \text{ maximal}.
\]
Assume that $S \subseteq V$ is simple and $U \subseteq W$ is maximal with $S \nsubseteq U$. Then $S \cap U \neq S$ and $S \subsetneq S+U$. Because $S$ is simple this implies $S \cap U = 0$, and because $U$ is maximal it implies $S+U = V$. So $V = S \oplus U$. This is a contradiction to the assumption that $V$ does not have a simple direct summand. So $S \subseteq U$ for all for all $S \subseteq V$ simple and all $U \subseteq V$ maximal.





\section{}








\end{document}
